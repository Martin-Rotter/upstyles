\documentclass[
	iwona=true,									% bezpatkový font
	printversion=false,					% pro finální sazbu (nepoužijí se barevné odkazy atp.), knižní sazba
	joinlists=true,							% zobrazí seznamy (zkratek, obrázků...) pod sebou
	glossaries=true,						% povolí podporu zkratek (seznam zkratek, deklarace zkratek)
	figures=true,								% povolí seznam obrázků
	tables=true,								% povolí seznam tabulek
	sourcecodes=true,						% povolí seznam zdr. kódů
	theorems=true,							% povolí seznam vět a teorémů
	bibencoding=utf8,						% kódování souboru s citacemi
	language=czech,							% jazyk práce
	encoding=utf8,							% kódování tohoto a vložených souborů
	master=true,								% mód pro diplomovou práci (Mgr.)
        field=inf,
  index=true									% chceme používat práci s rejstříkováním
]{updiplom}

\title{Dokonalá práce}				% název práce
\subtitle{Opravdu dokonalá} % volitelný podnázev práce
\title[english]{Perfect thesis}
\subtitle[english]{Really perfect}
\yearofsubmit{\the\year}			% rok odevzdání práce (při zakomentování se použije aktuální rok)
\author{Martin Rotter Albatros Pelikán}
\supervisor{Doc. Ing. Lenka Carr-Motyčková, CSc.}

\annotation{Tento dokument je fajn. Tento dokument je fajn. Tento dokument je fajn. Tento dokument je fajn. Tento dokument je fajn. Tento dokument je fajn. Tento dokument je fajn.}												% anotace práce - obvykle jeden odstavec textu
\keywords{dokument; fajn} % klíčová slova, oddělená středníkem
\thanks{Děkuji všem. Hlavně sobě.} % volitelné poděkování
\supplements{jedno kulaté CD/DVD} % přílohy

%% aktivuje bibliografii s daným souborem
\bibliography{bibliografie.bib}

\usepackage{multirow}
\usepackage{ltxdockit}
\usepackage{lipsum}

\begin{document}

% tiskne titulní stranu
\maketitle

\section{Úvod}
\lipsum{5-8}
Koukněte na můj \href{http://www.seznam.cz}{web}. Nebo na můj \mail{aaa@bbb.com}{e-mail} a navíc něco přes \hyperlink{http://www.abc.cz}{web ABC}.

Toto je můj klikací \index{indexovaný záznam}.

\subsection{Matematika}
$$\langle f \rangle, \lfloor g \rfloor,
\lceil h \rceil, \ulcorner i \urcorner$$

$$\left\{\frac{x^2}{y^3}\right\}$$

$$
A_{m,n} =
 \begin{pmatrix}
  a_{1,1} & a_{1,2} & \cdots & a_{1,n} \\
  a_{2,1} & a_{2,2} & \cdots & a_{2,n} \\
  \vdots  & \vdots  & \ddots & \vdots  \\
  a_{m,1} & a_{m,2} & \cdots & a_{m,n}
 \end{pmatrix}
$$

$$
M = \begin{bmatrix}
       \frac{5}{6} & \frac{1}{6} & 0           \\[0.3em]
       \frac{5}{6} & 0           & \frac{1}{6} \\[0.3em]
       0           & \frac{5}{6} & \frac{1}{6}
     \end{bmatrix}
$$

\section{Další sekce}
\lipsum{5-8}

\begin{lemma}[Démonické lemma]
Naše nové lemma.
\end{lemma}

\begin{align}
2+489 \\
7+2
\end{align}


\begin{table}
\begin{tabular}{c | c}
aaa & bbb 
\end{tabular}
\caption{Krutá tabulka}
\end{table}


\newacronym{UPOL}{UPOL}{Univerita Palackého}

\begin{definition}[Název definice]
Abcd. Abcd. Abcd. Abcd. Abcd. Abcd. Abcd. Abcd. Abcd. Abcd. Abcd. Abcd. Abcd. Abcd. Abcd. Abcd. Abcd. Abcd. Abcd. Abcd. Abcd. Abcd. Abcd. Abcd. Abcd. Abcd. Abcd. Abcd. Abcd. Abcd. \gls{UPOL}
\end{definition}

\begin{proof}[Název důkazu]
Abcd. Abcd. Abcd. Abcd. Abcd. Abcd. Abcd. Abcd. Abcd. Abcd. Abcd. Abcd. Abcd. Abcd. Abcd. Abcd. Abcd. Abcd. Abcd. Abcd. Abcd. Abcd. Abcd. Abcd. Abcd. Abcd. Abcd. Abcd. Abcd. Abcd. 
\end{proof}

\begin{remark}[Pumpovací věta]
Abcd. Abcd. Abcd. Abcd. Abcd. Abcd. Abcd. Abcd. Abcd. Abcd. Abcd. Abcd. Abcd. Abcd. Abcd. Abcd. Abcd. Abcd. Abcd. Abcd. Abcd. Abcd. Abcd. Abcd. Abcd. Abcd. Abcd. Abcd. Abcd. Abcd. 
\end{remark}

\begin{example}[Pumpovací věta]
Abcd. Abcd. Abcd. Abcd. Abcd. Abcd. Abcd. Abcd. Abcd. Abcd. Abcd. Abcd. Abcd. Abcd. Abcd. Abcd. Abcd. Abcd. Abcd. Abcd. Abcd. Abcd. Abcd. Abcd. Abcd. Abcd. Abcd. Abcd. Abcd. Abcd. 
\end{example}

\begin{lemma}[Název definice]
Abcd. Abcd. Abcd. Abcd. Abcd. Abcd. Abcd. Abcd. Abcd. Abcd. Abcd. Abcd. Abcd. Abcd. Abcd. Abcd. Abcd. Abcd. Abcd. Abcd. Abcd. Abcd. Abcd. Abcd. Abcd. Abcd. Abcd. Abcd. Abcd. Abcd. 
\end{lemma}

\begin{consequence}[Název důkazu]
Abcd. Abcd. Abcd. Abcd. Abcd. Abcd. Abcd. Abcd. Abcd. Abcd. Abcd. Abcd. Abcd. Abcd. Abcd. Abcd. Abcd. Abcd. Abcd. Abcd. Abcd. Abcd. Abcd. Abcd. Abcd. Abcd. Abcd. Abcd. Abcd. 
\end{consequence}

\begin{theorem}[Pumpovací věta]
Abcd. Abcd. Abcd. Abcd. Abcd. Abcd. Abcd. Abcd. Abcd. Abcd. Abcd. Abcd. Abcd. Abcd. Abcd. Abcd. Abcd. Abcd. Abcd. Abcd. Abcd. Abcd. Abcd. Abcd. Abcd. Abcd. Abcd. Abcd. Abcd. Abcd. 
\end{theorem}


\begin{upcode}{cpp}{}{\cpp}
int main("cs acsa") // komentar
int main("cs acsa") // komentar
int main("cs acsa") // komentar
int main("cs acsa") // komentar
int main("cs acsa") // komentar
\end{upcode}

\begin{upcode}{JavaScript}{}{JS}
new object() // komentar
\end{upcode}

\begin{upcode}{csharp}{}{\csharp}
public static int main("cs acsa") // komentar
\end{upcode}

\begin{upcode}{SQL}{}{SQL}
SELECT * FROM table_1; /* komentar */
\end{upcode}

\begin{upcode}{TutorialD}{}{TutorialD}
table_1 AND table_2;
\end{upcode}

% tiskne český závěr práce
\begin{upconclusions}
Závěr práce v \uv{českém} jazyce.
\end{upconclusions}

% tiskne anglický závěr práce
\begin{upconclusions}[english]
Thesis conclusions written in \uv{english}.
\end{upconclusions}

% tiskne přílohy
\appendix
\section{První příloha}
Text první přílohy

\section{Druhá příloha}
Text druhé přílohy

\printglossary

% tiskne bibliografii
% bibliografie používá BibLaTeX
\nocite{*}							% Vytiskne i necitované zdroje.
\printbibliography

% Použijte direktní bibliografii, pokud chcete.
%\begin{thebibliography}{9}
%
%\bibitem{lamport94}
%   Leslie Lamport,
%   \emph{\LaTeX: A Document Preparation System}.
%   Addison Wesley, Massachusetts,
%   2nd Edition,
%   1994.
%\end{thebibliography}

% tiskne rejstřík
\printindex

\end{document}