%%%  Vzorový dokument a dokumentace ke stylu pro závěrečnou práci na KI UP
%%%  Copyright (C) 2012 Martin Rotter, <rotter.martinos@gmail.com>
%%%  Copyright (C) 2014 Jan Outrata, <jan.outrata@upol.cz>

%%%  Pro získání PDF souboru dokumentu je třeba tento zdrojový text v
%%%  LaTeXu přeložit (dvakrát) programem pdfLaTeX a poté, v případě
%%%  použití programu BibLaTeX pro tvorbu seznamu literatury spustit
%%%  program Biber s parametrem jméno souboru zdrojového textu bez
%%%  přípony, v případě tvorby rejstříku přeložit vygenerovaný soubor
%%%  .idx programem Makeindex a v případě tvorby seznamu zkratek
%%%  spustit program Makeglossaries s parametrem jméno souboru
%%%  zdrojového textu bez přípony a následně opět (dvakrát) přeložit
%%%  zdrojový text programem pdfLaTeX.

\documentclass[
  master=false,               % Mód pro diplomovou práci (Mgr.).
  font=sans,                  % Bezpatkový font.
  printversion=false,         % Pro finální sazbu (nepoužijí se barevné odkazy atp.), knižní sazba.
  joinlists=true,             % Zobrazí seznamy (zkratek, obrázků...) pod sebou.
  glossaries=true,            % Povolí podporu zkratek (seznam zkratek, deklarace zkratek).
  figures=true,               % Povolí seznam obrázků.
  tables=true,                % Povolí seznam tabulek.
  sourcecodes=true,					  % Povolí seznam zdr. kódů.
  theorems=true,						  % Povolí seznam vět a teorémů.
  bibencoding=utf8,           % Kódování souboru s citacemi.
  language=czech,             % Jazyk práce.
  encoding=utf8,              % Kódování tohoto a vložených souborů.
  field=inf,                  % Studovaný obor (další možné hodnoty: uvt, binf).
  index=true,                 % Chceme používat práci s rejstříkováním.
  biblatex=true               % Chceme sázet bibliografii přes balík BibLaTeX.
]{updiplom}

\title{Dokonalá práce}                % Název práce.
\subtitle{Opravdu dokonalá}           % Volitelný podnázev práce.

\title[english]{Perfect thesis}       % Název práce (anglicky).
\subtitle[english]{Really perfect}    % Volitelný podnázev práce (anglicky).
\yearofsubmit{\the\year}			        % Rok odevzdání (při zakomentování se použije aktuální rok).

\author{Martin Rotter}
\supervisor{doc. Ing. Martin Vilém}

\annotation{Tento dokument je fajn. Tento dokument je fajn. Tento dokument je fajn. Tento dokument je fajn. Tento dokument je fajn. Tento dokument je fajn. Tento dokument je fajn.}                    % Anotace, jeden odstavec textu.
\annotation[english]{This document is good. This document is
  good. This document is good. This document is good. This document is
  good. This document is good. This document is good.} % Anotace práce anglicky - obvykle překlad anotace česky.

\keywords{dokument; fajn}             % Klíčová slova oddělená středníkem.
\keywords[english]{document; good}

\thanks{Děkuji všem. Hlavně sobě.}    % Volitelné poděkování.
\supplements{jedno kulaté CD/DVD}     % Popis příloh.

\bibliography{bibliografie.bib}       % Aktivuje bibliografii přes systém Biber/BibLaTex. Toto je třeba zakomentovat při použití manuální sazby bybliografie.

% Tyto balíčky jsou zde pouze pro účely tohoto dokumentu. Práce je nemusí obsahovat.
\usepackage{multirow}
\usepackage{ltxdockit}
\usepackage{lipsum}

\begin{document}
% Tiskné úvodní stránku, seznamy, anotaci, poděkování atp.
\maketitle

\section{Styly pro psaní bakalářských a diplomových prací}
Toto jsou styly pro psaní bakalářských a diplomových prací přes typografický systém \LaTeX{}, tedy \textbf{upstyles}.

\subsection{Požadavky a podprovaná prostředí}
Sada balíku \textbf{upstyles} podporuje následující distribuce systému \LaTeX{}:
\begin{itemize}
\item \TeX{} Live.
\end{itemize}

Jsou podporovány všechny výstupní ovladače, tedy jak \textbf{dvi}, tak \textbf{pdf} i \textbf{ps}. Funkčnost zmiňovaných distribucí byla ověřena na několika operačních systémech, mezi které patří:
\begin{enumerate}
\item Windows $8.1$,
\item Archlinux,
\item Debian.
\end{enumerate}

Důrazně se doporučuje používat aktuální verzi dané distribuce systému \LaTeX{}.


\subsection{Přepínače}
Styl updiplom je z hlediska uživatele zastoupen ekvivalentně nazvanou třídou, kterou je třeba volat na záčátku dokumentu:
\begin{upcode}{TeX}{}{Volání třídy \textbf{updiplom}}
\documentclass[
  master=true,
  font=sans,
  printversion=false,
  joinlists=true,
  glossaries=true,
  figures=true,
  tables=true,
  sourcecodes=true,
  theorems=true,
  bibencoding=utf8,
  language=czech,
  encoding=utf8,
  field=inf,
  index=true,
  biblatex=true
]{updiplom}
\end{upcode}

Následuje přehled přepínačů, je vždy uvedeno jméno přepínač, včetně výchozí hodnoty. Přepínače uvádí tabulka \ref{tab:prepinace}.

\begin{table}
\begin{center}
\caption{Seznam přepínačů}\label{tab:prepinace}
\scalebox{0.95}{\begin{tabular}{>{\bfseries}l >{\ttfamily}c L{8cm}}
{\normalfont Přepínač} & {\normalfont Výchozí hodnota} & {\normalfont Popis} \\
\hline
master & false & Povolí nebo zakáže režim diplomové práce. Výchozí režim je tedy bakalářská práce. \\

field & ainfp & Specifikuje studijní obor:\newline
\begin{description}
\item[ainf] Aplikovaná informatika\,--\,prezenční,
\item[ainfk] Aplikovaná informatika\,--\,kombinovaná,
\item[inf] Informatika\,--\,prezenční,
\item[infv] Informatika ve vzdělávání\,--\,kombinovaná,
\item[binf] Bioinformatika\,--\,prezenční.
\end{description} \\

font & serif & Zapne či vypne podporu pěkného bezpatkového fontu. Možné hodnoty jsou:\newline
\begin{description}
\item[sans] Bezpatkové písmo (písmo Iwona).
\item[serif] Patkové písmo (písmo Computer Modern).
\end{description} \\

encoding & utf8 & Kódování souboru dokumentu, doporučuje se ponechat výchozí hodnotu. \\

bibencoding & utf8 & Kódování souboru bibliografie. Tato volba má smysl pouze, pokud je použita bibliografie skrze balíček Bib\LaTeX{}. \\

language & czech & Jazyk hlavní práce. \\

printversion & false & Je-li zapnuto, pak budou odkazy vysázeny optimalizovaně pro knižní sazbu. Tuto volbu je nutno použít pro tisk práce. \\

joinlists & true & Je-li zapnuto, pak seznamy obrázků, tabulek či zdrojových kódů nebudou rozděleny na samostatné stránky. \\

figures & true & Je-li zapnuto, pak v seznamech položek bude zahrnut seznam obrázků. \\

tables & true & Je-li zapnuto, pak v seznamech položek bude zahrnut seznam tabulek. \\

theorems & false & Je-li zapnuto, pak v seznamech bude zahrnut seznam teorémů. \\

sourcecodes & false & Je-li zapnuto, pak v seznamech bude zahrnut seznam zdrojových kódů. \\

glossaries & false & Je-li zapnuto, pak na konci dokumentu bude vysázen seznam zkratek. \\

index & false & Zapíná podporu sazby rejstříku. \\

biblatex & true & Zapne sazbu bibliografie přes balík Bib\LaTeX{}.
\end{tabular}}
\end{center}
\end{table}

\subsection{Geometrie stránky}
Tento styl používá list velikosti $A4$. Pro sazbu prací je třeba použít jednostrannou sazbu. Levý okraj je rozšířen s ohledem na vazbu výsledné knižní podoby práce.

\section{Sazba částí dokumentu}
\subsection{Sazba úvodní strany či obsahu}
Vysázení všech podstatných částí úvodu práce obstará makro \upinlinecode{TeX}{!}{\\maketitle}. Pro správné vysázení všech částí a meta-informací je potřeba použí makra \upinlinecode{TeX}{!}{\\title}, \upinlinecode{TeX}{!}{\\author} a další. Jejich přehled lze najít ve zdrojovém souboru tohoto dokumentu. V případě použítí \textbf{pdf} výstupu se generuje i dodatečná hlavička souboru s meta-informacemi jako je autor dokumentu, název práce či dalšími.

\subsection{Závěry}
Závěr práce by se měl poskytnout ja v původním (českém jazyce), tak v jazyce anglickém. Pro sazbu závěru jsou k dispozici příslušná makra. Berte na vědomí, že v anglickém závěru se aktivuje plně anglická sazba se všemi konvencemi. Tedy je třeba používat anglické uvozovky a další správné typografické prvky.

\begin{upcode}{TeX}{}{Sazba závěrů}
% Tiskne český závěr práce.
\begin{upconclusions}
Závěr práce v \uv{českém} jazyce.
\end{upconclusions}

% Tiskne anglický závěr práce.
\begin{upconclusions}[english]
Thesis conclusions written in \uv{English}.
\end{upconclusions}
\end{upcode}

\subsection{Matematika}
Pro sazbu matematiky je k dispozici sada standardních maker.
$$\langle f \rangle, \lfloor g \rfloor,
\lceil h \rceil, \ulcorner i \urcorner$$

$$\left\{\frac{x^2}{y^3}\right\}$$

$$
A_{m,n} =
 \begin{pmatrix}
  a_{1,1} & a_{1,2} & \cdots & a_{1,n} \\
  a_{2,1} & a_{2,2} & \cdots & a_{2,n} \\
  \vdots  & \vdots  & \ddots & \vdots  \\
  a_{m,1} & a_{m,2} & \cdots & a_{m,n}
 \end{pmatrix}
$$

$$
M = \begin{bmatrix}
       \frac{5}{6} & \frac{1}{6} & 0           \\[0.3em]
       \frac{5}{6} & 0           & \frac{1}{6} \\[0.3em]
       0           & \frac{5}{6} & \frac{1}{6}
     \end{bmatrix}
$$

\subsection{Sazba literatury}
Pro sazbu literatury má uživatel dvě možnosti. Může použít služeb balíků Bib\LaTeX{}, který je pro \textbf{upstyles} zapnutý, či lze použít manuální sazbu bibliografie.
\subsubsection{Sazba bibliografie přes Bib\LaTeX{}}
Při použití tohoto balíku se data o použité literatuře ukládají do dedikovaného textového souboru, ukázku najdete i v tomto stylu pod jménem \upinlinecode{text}{!}{bibliografie.bib}.

Formát daného souboru je nad rámec této dokumentace a je na každém uživateli, aby si jej nastudoval. Bibliografie se tiskne makrem \upinlinecode{TeX}{!}{\\printbibliography}. Taktéž v preambuli dokumentu je třeba definovat, který soubor data bibliografie obsahuje, tedy například \upinlinecode{TeX}{!}{\\bibliography\{bibliografie.bib\}}.

Dokument, který využívá Bib\LaTeX{} je následně nutné přeložit jak pomocí překladače zvoleného ovladače, tak pomocí aplikace \upinlinecode{text}{!}{biber}. Více informací poskytne soubor \upinlinecode{text}{!}{Makefile} z distribuce tohoto stylu.

Výhodou tohoto přístupu je, že bibliografie se vysází automaticky a (obvykle) není třeba manuální úprava formátování.

\subsubsection{Manuální sazba bibliografie}
Manuální sazba obnáší vysázení prostředí \upinlinecode{text}{!}{thebibliography} ručně. To je nad rámec tohoto dokumentu. Ukázku tohoto přístupu lze samozřejmě nalézt ve zdrojovém souboru tohoto dokumentu nebo také \href{http://www.math.uiuc.edu/~hildebr/tex/bibliographies.html}{zde}.

Pro aktivaci manuální sazby bibliografie je třeba volat třídu \upinlinecode{text}{!}{updiplom} s parametrem \upinlinecode{text}{!}{biblatex=false}. Mějte, prosím, na paměti, že v tomto módu jsou makra \upinlinecode{text}{!}{\\bibliography} a \upinlinecode{text}{!}{\\printbibliography} nedostupná.

\subsection{Drobná makra}
Základní styl definuje hned několik maker pro usnadnění práce. Například makro \upinlinecode{TeX}{!}{\\buno} vysází řetezec \uv{bez újmy na obecnosti}. Je k dispozici i verze s prvním velkým písmenem, \upinlinecode{TeX}{!}{\\Buno}.

Je rovněž možno přidávat položky do seznamu zkratek. K tomu slouží makro \upinlinecode{TeX}{!}{\\newacronym}, které lze použít například jednoduše jako \upinlinecode{TeX}{!}{\\newacronym\{UPOL\}\{UPOL\}\{Univerita Palackého\}}. Na danou zkratku se pak lze odkazovat jednoduše, \upinlinecode{TeX}{!}{\\gls\{UPOL\}}.

Sazba uvozovek respektuje nastavení částí dokumentu, a proto se doporučuje používat makro \upinlinecode{TeX}{!}{\\uv}. V anglické závěru práce toto platí taky, viz tato PDF ukázka.

Styl podporuje sazbu odstavců v tabulkách, více obsahuje tabulka \ref{tab:odstavce}.

\begin{table}
\begin{center}
\caption{Seznam přepínačů}\label{tab:odstavce}
\begin{tabular}{L{4cm}|R{4cm}|L{4cm}}
\lipsum[23] & \lipsum[22] & \lipsum[21]
\end{tabular}
\end{center}
\end{table}

K dispozici jsou také makra pro sazbu \csharp{} (\upinlinecode{TeX}{!}{\\csharp}) či \cpp{} (\upinlinecode{TeX}{!}{\\cpp}).

\subsection{Sazba rejstříku}
Sazba rejstříku sestává z několika kroku:

\begin{enumerate}
\item Je třeba přes volbu \upinlinecode{TeX}{!}{index=true} rejstříkování povolit.
\item Použítím makra \upinlinecode{TeX}{!}{\\index} rejstříkovat vybrané pojmy.
\item Kompilovat s použitím utility \upinlinecode{TeX}{!}{makeindex}. Pro specifika tohoto kroku si stačí prohlédnout soubor \upinlinecode{text}{!}{Makefile}.
\end{enumerate}

Makro \upinlinecode{TeX}{!}{\\index} je redefinováno tak, že sází klikací odkaz na výraz v rejstříku. Je doporučeno jej použít ihned za výrazem\index{výraz}.

\textbf{Omezení redefinovaného makra \upinlinecode{TeX}{!}{\\index}}: klikací odkaz nefunguje, pokud použijete konstrukci \upinlinecode{TeX}{!}{\\index\{výraz|makro\}} (resp. \upinlinecode{TeX}{!}{\\index\{výraz|(makro\}}), např. \upinlinecode{TeX}{!}{\\index\{výraz|textit\}}.
  
Rejstřík lze vysázet pomocí makra \upinlinecode{TeX}{!}{\\printindex}.

\begin{lemma}[Démonické lemma]
Naše nové lemma~\cite{kniha2}.
\end{lemma}

\begin{align}
2+489 \\
7+2
\end{align}

\newacronym{UPOL}{UPOL}{Univerita Palackého}

\begin{definition}[Název definice]
Abcd. Abcd. Abcd. Abcd. Abcd. Abcd. Abcd. Abcd. Abcd. Abcd. Abcd. Abcd. Abcd. Abcd. Abcd. Abcd. Abcd. Abcd. Abcd. Abcd. Abcd. Abcd. Abcd. Abcd. Abcd. Abcd. Abcd. Abcd. Abcd. Abcd. \gls{UPOL}
\end{definition}

\begin{proof}[Název důkazu]
Abcd. Abcd. Abcd. Abcd. Abcd. Abcd. Abcd. Abcd. Abcd. Abcd. Abcd. Abcd. Abcd. Abcd. Abcd. Abcd. Abcd. Abcd. Abcd. Abcd. Abcd. Abcd. Abcd. Abcd. Abcd. Abcd. Abcd. Abcd. Abcd. Abcd. 
\end{proof}

\begin{remark}[Pumpovací věta]
Abcd. Abcd. Abcd. Abcd. Abcd. Abcd. Abcd. Abcd. Abcd. Abcd. Abcd. Abcd. Abcd. Abcd. Abcd. Abcd. Abcd. Abcd. Abcd. Abcd. Abcd. Abcd. Abcd. Abcd. Abcd. Abcd. Abcd. Abcd. Abcd. Abcd. 
\end{remark}

\begin{example}[Pumpovací věta]
Abcd. Abcd. Abcd. Abcd. Abcd. Abcd. Abcd. Abcd. Abcd. Abcd. Abcd. Abcd. Abcd. Abcd. Abcd. Abcd. Abcd. Abcd. Abcd. Abcd. Abcd. Abcd. Abcd. Abcd. Abcd. Abcd. Abcd. Abcd. Abcd. Abcd. 
\end{example}

\begin{lemma}[Název definice]
Abcd. Abcd. Abcd. Abcd. Abcd. Abcd. Abcd. Abcd. Abcd. Abcd. Abcd. Abcd. Abcd. Abcd. Abcd. Abcd. Abcd. Abcd. Abcd. Abcd. Abcd. Abcd. Abcd. Abcd. Abcd. Abcd. Abcd. Abcd. Abcd. Abcd. 
\end{lemma}

\begin{consequence}[Název důkazu]
Abcd. Abcd. Abcd. Abcd. Abcd. Abcd. Abcd. Abcd. Abcd. Abcd. Abcd. Abcd. Abcd. Abcd. Abcd. Abcd. Abcd. Abcd. Abcd. Abcd. Abcd. Abcd. Abcd. Abcd. Abcd. Abcd. Abcd. Abcd. Abcd. 
\end{consequence}

\begin{theorem}[Pumpovací věta]
Abcd. Abcd. Abcd. Abcd. Abcd. Abcd. Abcd. Abcd. Abcd. Abcd. Abcd. Abcd. Abcd. Abcd. Abcd. Abcd. Abcd. Abcd. Abcd. Abcd. Abcd. Abcd. Abcd. Abcd. Abcd. Abcd. Abcd. Abcd. Abcd. Abcd. 
\end{theorem}


\begin{upcode}{cpp}{}{\cpp}
int main("cs acsa") // komentar
int main("cs acsa") // komentar
int main("cs acsa") // komentar
int main("cs acsa") // komentar
int main("cs acsa") // komentar
\end{upcode}

\begin{upcode}{JavaScript}{}{JS}
new object() // komentar
\end{upcode}

\begin{upcode}{csharp}{}{\csharp}
public static int main("cs acsa") // komentar
\end{upcode}

\begin{upcode}{SQL}{}{SQL}
SELECT * FROM table_1; /* komentar */
\end{upcode}

\begin{upcode}{TutorialD}{}{TutorialD}
table_1 AND table_2;
\end{upcode}

% tiskne český závěr práce
\begin{upconclusions}
Závěr práce v \uv{českém} jazyce.
\end{upconclusions}

% tiskne anglický závěr práce
\begin{upconclusions}[english]
Thesis conclusions written in \uv{english}.
\end{upconclusions}

% tiskne přílohy
\appendix
\section{První příloha}
Text první přílohy

\section{Druhá příloha}
Text druhé přílohy

\printglossary

% tiskne bibliografii
% bibliografie používá BibLaTeX
\nocite{*}							% Vytiskne i necitované zdroje.
\printbibliography

% Použijte direktní bibliografii, pokud chcete.
%\begin{thebibliography}{99}
%\bibitem{kniha2} \uppercase{Hawke}, Paul. NanoHttpd: Light-weight HTTP server designed for embedding in other applications. GitHub [online]. 2014-05-12. [cit. 2014-12-06]. Dostupné z: \url{https://github.com/NanoHttpd/nanohttpd}
%
%\bibitem{jeske13} \uppercase{Jeske}, David; \uppercase{Novák}, Josef. Simple HTTP Server in \csharp: Threaded synchronous HTTP Server abstract class, to respond to HTTP requests. CodeProject: For those who code [online]. 2014-05-24. [cit. 2014-12-06]. Dostupné z: \url{http://www.codeproject.com/Articles/137979/Simple-HTTP-Server-in-C}
%
%\bibitem{uzis2012} \uppercase{ÚSTAV ZDRAVOTNICKÝCH INFORMACÍ A STATISTIKY ČR}. Lékaři, zubní lékaři a farmaceuti 2012 [online]. Praha 2, Palackého náměstí 4: Ústav zdravotnických informací a statistiky ČR, 2012 [cit. 2014-12-06]. ISBN 978-80-7472-089-5. Dostupné z: \url{http://www.uzis.cz/publikace/lekari-zubni-lekari-farmaceuti-2012}
%\end{thebibliography}

% tiskne rejstřík
\printindex

\end{document}