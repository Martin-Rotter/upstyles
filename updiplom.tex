\documentclass[12pt]{article}

\usepackage[
	printversion=true,								% pro finální sazbu (nepoužijí se barevné odkazy atp.), knižní sazba
	%joinlists=false,							% zobrazí seznamy (zkratek, obrázků...) pod sebou
	glossaries=true,						% povolí podporu zkratek (seznam zkratek, deklarace zkratek)
	%figures=true,							% povolí seznam obrázků
	%tables=true,							% povolí seznam tabulek
	sourcecodes=true,						% povolí seznam zdr. kódů
	bibencoding=utf8,						% kódování souboru s citacemi
	language=czech,
	encoding=utf8,							% kódování tohoto a vložených souborů
	master									% píšeme magisterskou (diplomovou) práci
]{updiplom}

\title{Dokonalá práce}						% název práce
\subtitle{Opravdu dokonalá}
\yearoffinish{\the\year}
\author{Martin Rotter}
\class{APLINF, III. ročník}
\supervisor{Mgr. Petr Velký}

\annotation{Tento dokument je fajn. Tento dokument je fajn. Tento dokument je fajn.
Tento dokument je fajn. Tento dokument je fajn.
 Tento dokument je fajn. Tento dokument je fajn.}			% anotace práce - obvykle jeden odstavec textu
\thanks{Děkuji všem. Hlavně sobě.}						% poděkování


%% aktivuje bibliografii s daným souborem
\bibliography{bibliography.bib}

\usepackage{multirow}
\usepackage{ltxdockit}
\usepackage{lipsum}

\begin{document}

% tiskne titulní stranu
\maketitle

\section{Úvod}
\lipsum{5-8}

\section{Další sekce}
\lipsum{5-8}

\begin{lemma}[Démonické lemma]
Naše nové lemma.
\end{lemma}

\begin{align}
2+489 \\
7+2
\end{align}


\begin{table}
\caption{Krutá tabulka}
\begin{tabular}{c | c}
aaa & bbb 
\end{tabular}
\end{table}


\newacronym{UPOL}{UPOL}{Univerita Palackého}

\begin{definition}[Název definice]
Abcd. Abcd. Abcd. Abcd. Abcd. Abcd. Abcd. Abcd. Abcd. Abcd. Abcd. Abcd. Abcd. Abcd. Abcd. Abcd. Abcd. Abcd. Abcd. Abcd. Abcd. Abcd. Abcd. Abcd. Abcd. Abcd. Abcd. Abcd. Abcd. Abcd. \gls{UPOL}
\end{definition}

\begin{prooof}[Název důkazu]
Abcd. Abcd. Abcd. Abcd. Abcd. Abcd. Abcd. Abcd. Abcd. Abcd. Abcd. Abcd. Abcd. Abcd. Abcd. Abcd. Abcd. Abcd. Abcd. Abcd. Abcd. Abcd. Abcd. Abcd. Abcd. Abcd. Abcd. Abcd. Abcd. Abcd. 
\end{prooof}

\begin{remark}[Pumpovací věta]
Abcd. Abcd. Abcd. Abcd. Abcd. Abcd. Abcd. Abcd. Abcd. Abcd. Abcd. Abcd. Abcd. Abcd. Abcd. Abcd. Abcd. Abcd. Abcd. Abcd. Abcd. Abcd. Abcd. Abcd. Abcd. Abcd. Abcd. Abcd. Abcd. Abcd. 
\end{remark}

\begin{lemma}[Název definice]
Abcd. Abcd. Abcd. Abcd. Abcd. Abcd. Abcd. Abcd. Abcd. Abcd. Abcd. Abcd. Abcd. Abcd. Abcd. Abcd. Abcd. Abcd. Abcd. Abcd. Abcd. Abcd. Abcd. Abcd. Abcd. Abcd. Abcd. Abcd. Abcd. Abcd. 
\end{lemma}

\begin{consequence}[Název důkazu]
Abcd. Abcd. \index{Abcd}. Abcd. Abcd. Abcd. Abcd. Abcd. Abcd. Abcd. Abcd. Abcd. Abcd. Abcd. Abcd. Abcd. Abcd. Abcd. Abcd. Abcd. Abcd. Abcd. Abcd. Abcd. Abcd. Abcd. Abcd. Abcd. Abcd. Abcd. 
\end{consequence}

\begin{theorem}[Pumpovací věta]
Abcd. Abcd. Abcd. Abcd. Abcd. Abcd. Abcd. Abcd. Abcd. Abcd. Abcd. Abcd. Abcd. Abcd. Abcd. Abcd. Abcd. Abcd. Abcd. Abcd. Abcd. Abcd. Abcd. Abcd. Abcd. Abcd. Abcd. Abcd. Abcd. Abcd. 
\end{theorem}

\begin{example}[Pumpovací věta]
Abcd. Abcd. Abcd. Abcd. Abcd. Abcd. Abcd. Abcd. Abcd. Abcd. Abcd. Abcd. Abcd. Abcd. Abcd. Abcd. Abcd. Abcd. Abcd. Abcd. Abcd. Abcd. Abcd. Abcd. Abcd. Abcd. Abcd. Abcd. Abcd. Abcd. 
\end{example}


\begin{upcode}{cpp}{}{Můj code}
int main("csacsa") // komentar
\end{upcode}

\upendofmainmatter

% tiskne český závěr práce
\begin{upconclusions}[czech]
Závěr práce v \uv{českém} jazyce.
\end{upconclusions}

% tiskne anglický závěr práce
\begin{upconclusions}[english]
Thesis conclusions written in \uv{english}.
\end{upconclusions}

% tiskne přílohy
\appendix
\section{První příloha}
Text první přílohy

\section{Druhá příloha}
Text druhé přílohy

\upmakeendlists

% tiskne bibliografii
% bibliografie používá BibLaTeX
\nocite{*}							% Vytiskne i necitované zdroje.
\printbibliography

% Použijte direktní bibliografii, pokud chcete.
%\begin{thebibliography}{9}
%
%\bibitem{lamport94}
%   Leslie Lamport,
%   \emph{\LaTeX: A Document Preparation System}.
%   Addison Wesley, Massachusetts,
%   2nd Edition,
%   1994.
%\end{thebibliography}

% tiskne rejstřík
\printindex

\end{document}