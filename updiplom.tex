\documentclass[a4paper,12pt]{article}

\usepackage[
	%printversion=true,								% pro finální sazbu (nepoužijí se barevné odkazy atp.), knižní sazba
	joinlists=false,							% zobrazí seznamy (zkratek, obrázků...) pod sebou
	%glossaries=false,						% povolí podporu zkratek (seznam zkratek, deklarace zkratek)
	%figures=true,							% povolí seznam obrázků
	%tables=true,							% povolí seznam tabulek
	%sourcecodes=false,						% povolí seznam zdr. kódů
	bibfile=bibliography.bib,				% soubor s citacemi (BibLaTeX)
	bibencoding=utf8,						% kódování souboru s citacemi
	language=czech,
	encoding=utf8,							% kódování tohoto a vložených souborů
	master									% píšeme magisterskou (diplomovou) práci
]{updiplom}

\uptitle{Dokonalá práce}						% název práce
\upsubtitle{Opravdu dokonalá}				% podtitulek práce - použijte \upsubtitle{\null} pokud nemáte v práci podnadpisek
\upyear{\the\year}
\upauthor{Martin Rotter}
\upclass{APLINF, III. ročník}
\upleader{Mgr. Petr Velký}
\upanot{Tento dokument je fajn. Tento dokument je fajn. Tento dokument je fajn.
Tento dokument je fajn. Tento dokument je fajn.
 Tento dokument je fajn. Tento dokument je fajn.}			% anotace práce - obvykle jeden odstavec textu
\upthanks{Děkuji všem. Hlavně sobě.}						% poděkování


%% aktivuje bibliografii s daným souborem
\bibliography{bibliography.bib}

\usepackage{multirow}
\usepackage{ltxdockit}
\usepackage{lipsum}

\begin{document}

% tiskne titulní stranu
\maketitle

% tiskne dvě strany s anotací a poděkováním
\upthanksanot

% tiskne obsah a seznamy obrázku, tabulek a případně zdr. kódů, více v makrech \uptableofcontents a \upprintlists
\uptocandlists

\section{Úvod}
\lipsum{5-8}

\section{Další sekce}
\lipsum{5-8}

\begin{uplemma}[Démonické lemma]
Naše nové lemma.
\end{uplemma}

\newacronym{UPOL}{UPOL}{Univerita Palackého}

\begin{uptheorem}[Název definice]
Naše nová \gls{UPOL} definice.
\end{uptheorem}

\begin{upproof}[Název důkazu]
Náš nový důkaz.
\end{upproof}

%% neplovoucí zdrojový kód
\begin{upcode}{Můj code}{}{SQL}
SELECT * FROM;
\end{upcode}


\begin{upquote}[Pumpovací věta]
Náš nový důkaz. \index{důkaz}
\end{upquote}

\upendofmainmatter

% tiskne český závěr práce
\begin{upconclusions}[czech]
Závěr práce v \enquote{českém} jazyce.
\end{upconclusions}

% tiskne anglický závěr práce
\begin{upconclusions}[english]
Thesis conclusions written in \enquote{english}.
\end{upconclusions}

% tiskne bibliografii
% bibliografie používá BibLaTeX
\nocite{*}							% Vytiskne i necitované zdroje.
\printbibliography

% Použijte direktní bibliografii, pokud chcete.
%\begin{thebibliography}{9}
%
%\bibitem{lamport94}
%   Leslie Lamport,
%   \emph{\LaTeX: A Document Preparation System}.
%   Addison Wesley, Massachusetts,
%   2nd Edition,
%   1994.
%\end{thebibliography}

% tiskne seznam zkratek
% seznam zkratek se netiskne před práci ale až nakonec
\printglossary

% tiskne seznam teorémů
\listoftheorems

% tiskne přílohy
\appendix
\section{První příloha}
Text první přílohy

\section{Druhá příloha}
Text druhé přílohy

% tiskne rejstřík
\printindex

\end{document}