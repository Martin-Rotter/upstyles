\documentclass[a4paper,12pt]{article}

\usepackage[
	abbreviations,
	figures,
	tables,
	listings,
	language=czech,
	bibfile=bibliography.bib,
	bibencoding=utf8,
	bibstyle=numeric,
	encoding=utf8,
	]{updiplomex}

\uptitle{Název práce}						% název práce
\upsubtitle{Podtitulek práce}				% podtitulek práce - použijte \upsubtitle{\null} pokud nemáte v práci podnadpisek
\upyear{\the\year}
\upauthor{Martin Rotter}
\upclass{APLINF, III. ročník}
\upleader{Mgr. Petr Velký}
\upanot{Tento dokument je fajn.}			% anotace práce - obvykle jeden odstavec textu
\upthanks{Děkuji všem.}						% poděkování

\usepackage{multirow}
\usepackage{ltxdockit}

\begin{document}

% tiskne titulní stranu
\upmaketitle

% tiskne dvě strany s anotací a poděkováním
\upthanksanot

% tiskne obsah a seznamy obrázku, tabulek a případně zdr. kódů, více v makrech \uptableofcontents a \upprintlists
\uptocandlists

\section{Styl updiplomex}
Styl \keyword{updiplomex} je novou verzí původních \LaTeX{} stylů pro psaní diplomových prací a nabízí některá vylepšení a další \enquote{finesy}.

\section{Podpora ovladačů a prostředí}
V současné době byla zahozena podpora \keyword{cslatex} a \keyword{pdfcslatex}. Ke kompilaci dokumentů je tedy používán výlučně \keyword{latex} a \keyword{pdflatex}. Toto použití má hned několik důvodů:
\begin{itemize}
\item není důvod obecně nepoužívat PDF jako přímý výstup (konverze PDF $\rightarrow$ PS jsou bezbolestné)
\item nepoužití cslatex-u přináší daší bonusy, jako například multijazykové mo-žnosti s balíkem \keyword{babel}
\item pdflatex je moderní
\end{itemize}

\section{Rozměry stránky A4 a mezery}
Rozložení textu na stránce ja nastaveno tak, aby splňovalo standardy. Horní mezera má $3$ cm, spodní $1.5$ cm, levá a pravá mezera mají shodně $2.5$ cm, pričemž k levé straně je přídán $1.5$ cm pro pevnou vazbu. Styl je určen pro jednostranný tisk.

\section{Možnosti balíku}
Balíku updiplomex je možno volat s několika vstupními argumenty, které mohou zásadním způsobem ovlivnit vzhled dokumentu. Každý argument je popsán v následujícím výčtu.

\begin{optionlist}
\boolitem[false]{iwona}
Je-li true, pak se použije okrasnější písmo Iwona, jinak výchozí.
\boolitem[false]{final}
Je-li true, pak se hypertextové odkazy nebudou obarvovat, což se hodí pro knižní tisk.
\boolitem[false]{blacklogo}
Je-li true, tak bude titulní strana obsahovat černé logo místo bílého.
\boolitem[false]{master}
Je-li true, tak bude vypsán na titulní straně text \enquote{Diplomová práce}.
\boolitem[false]{joinlists}
Je-li true, tak budou seznamy a obsah sdruženy za sebe bez vytváření nových stran. 
\boolitem[false]{figures}
Je-li true, tak budou seznamy obsahovat seznam obrázků.
\boolitem[false]{tables}
Je-li true, tak budou seznamy obsahovat seznam tabulek.
\boolitem[false]{abbreviations}
Je-li true, tak bude dostupné makro \verb|\upprintabbrevlist|, které tiskne seznam zkratek. Více v sekci o zkratkách.
\boolitem[false]{listings}
Je-li true, tak budou seznamy obsahovat seznam zdrojových kódů. 
\optitem[czech]{language}{\prm{řetězec}}
Nastavuje jazyk používaný balíkem babel. 
\optitem[bibliography.bib]{bibfile}{\prm{řetězec}}
Nastavuje soubor s bibliografií pro Bib\LaTeX  
\optitem[utf8]{bibencoding}{\prm{řetězec}}
Nastavuje kódování onoho souboru z předchozího řádku. 
\optitem[numeric]{bibstyle}{\prm{numeric,authoryear,iso-authoryear,iso-numeric\ldots}}
Nastavuje způsob citování dokumentů z bibliografie.

Tradiční způsob citování jest s využítím číselných citací, ale metoda authoryear se může jevit přehlednější. Metody iso-authoryear a iso-numeric jsou v experimentální fázi vývoje.

\optitem[Seznam zkratek]{abbrevtitle}{\prm{řetězec}}
Nastavuje nadpis pro seznam zkratek. 
\optitem[Anotace]{titleanot}{\prm{řetězec}}
Nastavuje nadpis pro anotaci.
\optitem[Poděkování]{titlethx}{\prm{řetězec}}
Nastavuje název PDF záložky pro poděkování.
\optitem[Seznam zdrojových kódů]{titlelistings}{\prm{řetězec}}
Nastavuje nadpis pro seznam zdr. kódů.
\optitem[Zdrojový kód]{namelistings}{\prm{řetězec}}
Nastavuje nadpisek pro každý zdr. kód, tohle využívá například \verb|\caption| a seznam zdr. kódů. 
\optitem[Závěr]{titleconclusioncz}{\prm{řetězec}}
Nastavuje nadpis pro český závěr.
\optitem[Conclusions]{titleconclusionen}{\prm{řetězec}}
Nastavuje nadpis pro anglický záver.
\optitem[Titulní strana]{titlebookmarktitlepage}{\prm{řetězec}}
Nastavuje nadpisek pro každý zdr. kód, tohle využívá například \verb|\caption| a seznam zdr. kódů. 
\optitem[utf8]{encoding}{\prm{řetězec}}
Nastavuje kódování vstupních tex souborů, tedy i tohoto souboru. 
\end{optionlist}

\section{Překlad ukázkového dokumentu}
Překlad dokumentů se provádí před pdflatex\index{pdflatex}, navíc je třeba použít makeindex a biber pro překlad rejstříku a bibliografie. Vše je obsaženo v souboru Makefile.

\begin{upcode}{Překlad dokumentů}{code:compilation}{make}
all: start guide finish

start:
	echo "Makefile has started its work."

guide:	updiplomex.tex
	pdflatex updiplomex
	pdflatex updiplomex
	makeindex updiplomex.idx -s index.ist
	makeglossaries updiplomex
	biber updiplomex
	pdflatex updiplomex
	pdflatex updiplomex

finish:
	echo "Makefile has finished its work."

clean:
	rm -f *.lo* *.aux *.ind *.idx *.ilg *.toc
\end{upcode}

\section{Další doplňky}
\subsection{Seznam zkratek}
Viditelnost seznamu zkratek ovlivňuje přepínač \keyword{abbreviations} a \keyword{abbrevtitle}. Seznam zkratek je vykreslován standardně s tím, že navíc jsou (oproti normálu) vykresleny zkratky tučně.

%% Deklarace zkratky
\upabbrevdeclare{UPOL}{UPOL}{Univerzita Palackého v Olomouci}
\newacronym{RS}{ŘS}{Řízení systémů}

Zkratky lze definovat kdekoliv v těle dokumentu použitím příkazu ve zdrojovém kódu \ref{code:abbrev}.

\begin{upcode}{Syntaxe pro tvorbu zkratky}{code:abbrev}{make}
\upabbrevdeclare{LABEL}{ZKRATKA}{TEXT ZKRATKY}
\end{upcode}
Odkazovat se na zkratky lze použitím makra \verb|\upabbrevref{LABEL}|. Pokud například máme definovanou zkratku UPOL, tak se na ní můžeme odkázat několikrát, třeba dvakrát, za sebou. Tedy \upabbrevref{UPOL} a ještě jednou \upabbrevref{UPOL}, dále také \upabbrevref{RS} a \upabbrevref{RS}.

\subsection{Teorémy}
V rámci stylu existují prostředí s jedním volitelným argumentem (nadpis) \keyword{uplemma}, \keyword{uptheorem}, \keyword{upquote} a \keyword{upproof}. Tyto prostředí slouží k sazbě důkazů, vět, definic nebo lemmat.

\begin{uplemma}[Démonické lemma]
Naše nové lemma.
\end{uplemma}

\begin{uptheorem}[Název definice]
Naše nová definice.
\end{uptheorem}

\begin{upproof}[Název důkazu]
Náš nový důkaz.
\end{upproof}

\begin{upquote}[Pumpovací věta]
Náš nový důkaz.
\end{upquote}

\subsection{Užitečná makra}
Balík obsahuje makra pro sázení názvů některých programovacích jazyků, prozatím tu máme makra \verb|\csharp| a \verb|\cpp|, která vykrelsují symboly \csharp a \cpp.

Dále je obsaženo například makro \verb|\buno|, které vypíše \enquote{\buno}. Makro je dostupné i s počátečním velkým písmenem jako \verb|\Buno|.

Rovněž výčty obsahují nový úvodní symbol, původní tečka byla nahrazena toutéž (avšak šedou) tečkou.
\begin{itemize}
\item šedá tečka
\end{itemize}

Navíc je zde makro \verb|\keyword|, které vypíše zadaný argument \keyword{šedou barvou v neproporcionálním fontu}.

\paragraph{Speciální odstavec}
Tohle je ukázkový odstavec, který co se týče jeho hierarchie v dokumentu podobný k \verb|\subsubsubsection|.

\section{Bibliografie}
Bibliografie používá balík Bib\LaTeX, který je na správu\footnote{Poznámka pod čarou.} bibliografie v současnosti nejlepší, navíc je implementován citační formát dle ISO690, byť tady jsou zatím viditelné problémy. Nicméně vše by mělo správně fungovat.

\begin{table}
\caption{Ukázková citace}
\begin{center}
\begin{tabular}{c|l}
Kód & Výsledek \\
\hline \\
\verb|\cite{nigel:csharp}| & \cite{nigel:csharp} \\
\verb|\citep{nigel:csharp}| & \citep{nigel:csharp} \\
\verb|\citet{nigel:csharp}| & \citet{nigel:csharp} \\
\verb|\citealt{nigel:csharp}| & \citealt{nigel:csharp}
\end{tabular}
\end{center}
\end{table}

\begin{upcode}{Ukázkový \cpp kód}{code:cpp}{C++}
int main(int argc, char *argv[]) {
	return 0;
}
\end{upcode}

\section{Zdrojové kódy}
Je použit balík listings. Pomocí něj lze zobrazovat pěkné zdrojové kódy. Ukázku vidíte například niže. Zatím je nastavena barevná syntaxe pro jazyk \cpp, ale podporována je spousta dalších jazyků.

% tiskne český závěr práce
\begin{upconclusions}[czech]
Závěr práce v \enquote{českém} jazyce.
\end{upconclusions}

% tiskne anglický závěr práce
\begin{upconclusions}[english]
Thesis conclusions written in \enquote{english}.
\end{upconclusions}

% tiskne seznam zkratek
% seznam zkratek se netiskne před práci ale až nakonec
\upprintabbrevlist

% tiskne seznam teorémů
\upprinttheoremlist

% tiskne bibliografii
% bibliografie používá BibLaTeX
\upprintbibliography

% tiskne přílohy
\upappendix
\section{První příloha}
Text první přílohy

\section{Druhá příloha}
Text druhé přílohy

% tiskne rejstřík
\upprintindex

\end{document}